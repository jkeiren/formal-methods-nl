\RequirePackage[hyphens]{url}
\documentclass[sigplan]{acmart}\settopmatter{printfolios=true,printccs=false,printacmref=false}

%% Conference information
%% Supplied to authors by publisher for camera-ready submission;
%% use defaults for review submission.
\acmConference[]{Formal Methods in the Netherlands}
\acmYear{2018}
\acmISBN{} % \acmISBN{978-x-xxxx-xxxx-x/YY/MM}
\acmDOI{} % \acmDOI{10.1145/nnnnnnn.nnnnnnn}
\startPage{1}

\setcopyright{none}
%\setcopyright{acmcopyright}
%\setcopyright{acmlicensed}
%\setcopyright{rightsretained}
%\copyrightyear{2018}           %% If different from \acmYear

%% Bibliography style
\bibliographystyle{ACM-Reference-Format}
%% Citation style
%\citestyle{acmauthoryear}  %% For author/year citations
%\citestyle{acmnumeric}     %% For numeric citations
%\setcitestyle{nosort}      %% With 'acmnumeric', to disable automatic
                            %% sorting of references within a single citation;
                            %% e.g., \cite{Smith99,Carpenter05,Baker12}
                            %% rendered as [14,5,2] rather than [2,5,14].
%\setcitesyle{nocompress}   %% With 'acmnumeric', to disable automatic
                            %% compression of sequential references within a
                            %% single citation;
                            %% e.g., \cite{Baker12,Baker14,Baker16}
                            %% rendered as [2,3,4] rather than [2-4].


%%%%%%%%%%%%%%%%%%%%%%%%%%%%%%%%%%%%%%%%%%%%%%%%%%%%%%%%%%%%%%%%%%%%%%
%% Note: Authors migrating a paper from traditional SIGPLAN
%% proceedings format to PACMPL format must update the
%% '\documentclass' and topmatter commands above; see
%% 'acmart-pacmpl-template.tex'.
%%%%%%%%%%%%%%%%%%%%%%%%%%%%%%%%%%%%%%%%%%%%%%%%%%%%%%%%%%%%%%%%%%%%%%


%% Some recommended packages.
\usepackage{booktabs}   %% For formal tables:
                        %% http://ctan.org/pkg/booktabs
\usepackage{subcaption} %% For complex figures with subfigures/subcaptions
                        %% http://ctan.org/pkg/subcaption

\hypersetup{breaklinks=true}

\begin{document}

%% Title information
\title{Practical Formal Methods}         %% [Short Title] is optional;
                                        %% when present, will be used in
                                        %% header instead of Full Title.

%% Author with single affiliation.
\author{Jeroen J.A. Keiren}
%\authornote{with author1 note}          %% \authornote is optional;
                                        %% can be repeated if necessary
\orcid{0000-0002-5772-9527}             %% \orcid is optional
\affiliation{
  %\position{Position1}
  \department{Department of Computer Science}              %% \department is recommended
  \institution{Open University of the Netherlands}            %% \institution is required
  %\streetaddress{Street1 Address1}
  %\city{Heerlen}
  %\state{State1}
  %\postcode{Post-Code1}
  %\country{The Netherlands}                       %% \country is recommended
}
\affiliation{
  %\position{Position2a}
  \department{Department of Digital Security, ICIS}             %% \department is recommended
  \institution{Radboud University}           %% \institution is required
  %\streetaddress{Street2a Address2a}
  \city{Nijmegen}
  %\state{State2a}
  %\postcode{Post-Code2a}
  \country{The Netherlands}                    %% \country is recommended
}

\affiliation{
  %\position{Position2a}
  %\department{}             %% \department is recommended
  \institution{Delft University of Technology}           %% \institution is required
  %\streetaddress{Street2a Address2a}
  \city{Delft}
  %\state{State2a}
  %\postcode{Post-Code2a}
  \country{The Netherlands}                   %% \country is recommended
}
\email{Jeroen.Keiren@ou.nl}          %% \email is recommended

% %% Abstract
% %% Note: \begin{abstract}...\end{abstract} environment must come
% %% before \maketitle command
% \begin{abstract}
% Text of abstract \ldots.
% \end{abstract}


% %% 2012 ACM Computing Classification System (CSS) concepts
% %% Generate at 'http://dl.acm.org/ccs/ccs.cfm'.
% \begin{CCSXML}
% <ccs2012>
% <concept>
% <concept_id>10011007.10011006.10011008</concept_id>
% <concept_desc>Software and its engineering~General programming languages</concept_desc>
% <concept_significance>500</concept_significance>
% </concept>
% <concept>
% <concept_id>10003456.10003457.10003521.10003525</concept_id>
% <concept_desc>Social and professional topics~History of programming languages</concept_desc>
% <concept_significance>300</concept_significance>
% </concept>
% </ccs2012>
% \end{CCSXML}

% \ccsdesc[500]{Software and its engineering~General programming languages}
% \ccsdesc[300]{Social and professional topics~History of programming languages}
% %% End of generated code


% %% Keywords
% %% comma separated list
% \keywords{keyword1, keyword2, keyword3}  %% \keywords are mandatory in final camera-ready submission


%% \maketitle
%% Note: \maketitle command must come after title commands, author
%% commands, abstract environment, Computing Classification System
%% environment and commands, and keywords command.
\maketitle

\section{Introduction}
Today, we heavily depend on software. We do not only use the computers on our
desktops and the mobile phones in our pockets. Financial infrastructures and
automatic stock trading are controlled by computers, and computer systems are
embedded in home appliances such as televisions, safety critical systems such as
cars and airplanes, as well as systems controlling (access to) infrastructure
such as bridges and tunnels.

During the development of software, inevitably, mistakes are made. In fact, on
average, every 1000 lines of code contain up to 10-16 bugs \cite{ostrand_distribution_2002,ostrand_where_2004}.
Some of these bugs will only show up once the system is running. At that
point, the consequences can range from being harmless -- e.g. needing to restart
your phone because it freezes --, to very severe -- such as a car crashing
\cite{ntsb_2018}, hundreds of millions of dollars being lost in in the stock markets
\cite{glitch_2012}. Furthermore, a growing reliance on battery-powered devices and the
effects of climate change have resulted in an increased interest in \emph{green
computing}. Bugs that lead to quick draining of batteries have gained a lot of
publicity \cite{facebook_bug_2017}.

To effectively detect or avoid such bugs early (preferably before software is
used) calls for a range of different tools and techniques. This ranges from
exact and exhaustive verification methods such as (automated) theorem proving
and model checking, to formal testing techniques such as model based testing,
automated testing at the level of graphical user interfaces, as well as more
traditional testing techniques such as unit- and integration testing.
The formal methods and software engineering communities in the Netherlands have
a long track record in developing such techniques.

When considering, especially, the more formal approaches such as model checking,
theorem proving and model-based testing, there is a lot of anecdotal evidence
that they are effective in finding and avoiding bugs, for instance
\cite{hwong_formalising_2013,Madlener10,fiterau_2016}. However, the techniques
typically require experts that are well-versed in both the application domain as
well as in the formal methods applied. Furthermore, a clear business case for
the application of formal methods is missing.

Besides the mainly theoretical research into the foundations of formal methods
based verification and testing techniques, there are three research directions
that deserve our undivided attention.
\begin{enumerate}
  \item Develop formal methods that can be used by software engineers that are not formal methods experts.
  \item Establish a business case for the industrial application of formal methods through empirical research.
  \item Determine how formal methods can best be integrated in software engineering and computer science curricula. 
\end{enumerate}
I will detail each of these three points in the rest of this abstract.

\section{Bringing formal methods to the masses}

The application of formal methods such as model checking or model based testing
to industrial cases is still very much an expert activity. Tools often rely on
models of the software, instead of the software itself, and the models are
specified in domain-specific languages whose syntax is far from the the
programming languages software engineers are used to. Throughout the years many
applications of such techniques have been reported as a success, e.g.,
\cite{Madlener10,fiterau_2016}, but large-scale
industrial application has yet to gain traction.

In software model-checking, some successes have been obtained, e.g., at
Microsoft using
SLAM\footnote{\url{https://www.microsoft.com/en-us/research/project/slam/}, accessed 9
August 2018} \cite{SLAM} and TERMINATOR, continued as T2 \cite{T2}, for the
verification of C code, where verifying a limited set of properties on an actual
(C-code) implementation seems to be one of the success criteria.

To bring formal methods to the masses, as a community we need to invest not just
in new techniques, but also in the engineering of tooling and languages that
can be used in the industrial software engineering process, and recognise that this step is not ``just trivial engineering''. We should therefore not stop
at the development of prototype tools, but also look at industry needs and 
invest in making the tooling mature enough for use in production systems. At CERN, for example, we went from high-level academic tools, down to an integration of the verification in the IDE used by the developers, effectively providing developers with push-button access to model checking \cite{hwong_formalising_2013}.

\section{Empirical evidence for the merits of formal methods}

Industrial application of formal methods are typically reported as succes stories.
However, from an industrial perspective, it is important to know the effects on
applying such techniques. How does the application of formal methods affect, for
example, time-to-market, number of bugs found after deployment, etc.

Unfortunately, such information is currently lacking from the formal methods
literature. Some vendors, such as Verum with its Dezyne toolkit -- which is based on mCRL2 \cite{dezyne_mcrl2} --, do report
benefits such as ``50\% reduction in development costs, 25\% reduction in the
cost of field defect and 20\% decrease in
time-to-market''\footnote{\url{https://www.verum.com}, accessed 9 August 2018.}, but the sources and reliability of such data are unclear.
In order to build a successful business case for the application of formal
methods in software engineering practice, a collaborative effort should be made
to collect data about the application of formal methods. Based on the collected
data, a business case for the application of formal methods should be developed. One possibility could be the collection of data in a large number of student projects, such as \cite{Rouwen2018}.


\section{Teaching formal methods}

A final advance in the acceptance of formal methods lies in the way we teach
formal methods. It appears commonplace in software engineering and computer
science curricula to present formal methods more as a research topic than as a
software engineering topic. A typical question one gets from students is ``so,
where and how are these techniques actually applied in industry''. In recent
years, I have seen more critical attitudes of students towards formal methods
courses. Should we do a better job at integrating formal methods into our
curricula, to really show the students what the can \emph{do with formal methods}
instead of \emph{how the formal methods work}?

Note that there is more to the teaching of formal methods. How can we
effectively teach such methods using modern tools and techniques such as massive
open online courses (MOOCs)? In particular, with (sometimes dramatically)
increasing numbers of students, and in the context of distance learning, we need
to find ways of providing feedback in formal methods education that does not
rely on expert feedback. Is it possible to build interactive online tools that
present feedback to students about their solutions? Are there more effective
ways of teaching formal methods than we currently use in our courses? Can we, in
this respect, learn from programming education research such as \cite{scratch,redesign}?


%% Bibliography
\bibliography{bibfile}

\end{document}
